\cleardoublepage
\chapter*{\titulos{RESUMEN}}
\addcontentsline{toc}{chapter}{\normalfont RESUMEN}
La presente investigación se propuso el diseño, la implementación y la evaluación de un motor adaptativo de evaluación educativa, orientado a la estimación precisa del nivel de conocimiento del estudiante y a la utilización de dicha estimación para personalizar el proceso de aprendizaje.

\vspace{0.3cm}

Desarrollar un motor adaptativo operativo que traduzca la evidencia de interacción del estudiante en decisiones pedagógicas fundamentadas, mediante la combinación de modelos psicométricos extensivamente validados IRT (Item Response Theory) y BKT/KT (Bayesian Knowledge Tracing / Knowledge Tracing) para medir con precisión el desempeño del estudiante y construir trayectorias de aprendizaje genuinamente personalizadas.

\vspace{0.3cm}

La metodología empleada se sustenta en un diseño de tipo cuantitativo y experimental, mediante la implementación de técnicas de simulación de estudiantes virtuales que permiten la generación de interacciones controladas y reproducibles. La recolección de datos se lleva a cabo de forma automática durante la ejecución del sistema, mediante registros de telemetría estructurada.

\vspace{0.3cm}

Para la evaluación del sistema se utilizaron métricas psicométricas y computacionales, destacando el Root Mean Squared Error (RMSE), el Mean Absolute Error (MAE), la convergencia de la habilidad latente ($\theta$) y la eficiencia adaptativa medida por el número de ítems requeridos. Los resultados muestran una reducción en la longitud de las evaluaciones y una estabilidad óptima del sistema ante escenarios con múltiples usuarios concurrentes.

\vspace{0.3cm}

Los hallazgos demuestran que el motor adaptativo propuesto constituye una solución efectiva para entornos educativos digitales.\\

\textbf{PALABRAS CLAVE:} Evaluación adaptativa, Estimación de habilidad latente, Modelos psicométricos adaptativos, Seguimiento probabilístico del aprendizaje, Simulación de estudiantes virtuales, Sistemas educativos inteligentes.
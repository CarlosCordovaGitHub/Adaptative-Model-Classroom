\cleardoublepage
\chapter*{\titulos{ABSTRACT}}
\addcontentsline{toc}{chapter}{\normalfont ABSTRACT}

This research focused on the design, implementation, and evaluation of an adaptive educational assessment engine aimed at accurately estimating students’ knowledge levels and using this estimation to personalize the learning process.

\vspace{0.3cm}
\vspace{0.3cm}

To develop an operational adaptive engine capable of translating evidence from student interactions into well-founded pedagogical decisions, through the combination of extensively validated psychometric models such as IRT (Item Response Theory) y BKT/KT (Bayesian Knowledge Tracing / Knowledge Tracing), in order to accurately measure student performance and construct genuinely personalized learning trajectories.

\vspace{0.3cm}

\vspace{0.3cm}

The methodology was based on a quantitative and experimental design, implemented through virtual student simulation techniques that allow the generation of controlled and reproducible interactions. Data collection was carried out automatically during system execution using structured telemetry records.

\vspace{0.3cm}

\vspace{0.3cm}

System evaluation employed both psychometric and computational metrics, including root mean square error (RMSE), mean absolute error (MAE), convergence of the latent ability parameter ($\theta$), and adaptive efficiency measured by the number of required items. Results indicate a reduction in assessment length and optimal system stability under scenarios with multiple concurrent users.

\vspace{0.3cm}

\vspace{0.3cm}

The findings demonstrate that the proposed adaptive engine represents an effective solution for digital educational environments.

\vspace{0.3cm}
\textbf{KEYWORDS:} Adaptive assessment, Adaptive psychometric models, Intelligent educational systems, Latent ability estimation, Probabilistic learning tracking, Virtual student simulation.